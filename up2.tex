// correcte sans function
app.post('/upload2', upload2.single('excelFile'), async (req, res) => {
  try {
      // Lecture du fichier Excel
      const workbook = xlsx.readFile(req.file.path);
      const sheetName = workbook.SheetNames[0];
      const sheet = workbook.Sheets[sheetName];
      const data = xlsx.utils.sheet_to_json(sheet, { header: 1 });
      const username = req.body.username;

      // Requête SQL pour insérer les données dans votre_table2
      const sql = `INSERT INTO votre_table2 (
          id_contrat, date_ouverture, date_cloture, type_offre, type_client, etat_ticket, 
          sujet, type_incident, indisponibilite, derangement, facturation, 
          internet_fixe, id_client, MOIS, Annee, userId	
      ) VALUES ?`;

      const values = data.slice(1).map(row => [...row.slice(0, 15), username]);

      await new Promise((resolve, reject) => {
          connection.query(sql, [values], (err, result) => {
              if (err) {
                  console.error('Erreur MySQL :', err.message);
                  return reject(err);
              }
              console.log('Données insérées avec succès :', result);
              resolve(result);
          });
      });

      // Calcul des KPI
      const mois = data[1][13];
      console.log('mois :', mois);
       //PRO
       const kpi3_11_pro_SDSL = data.slice(1).filter(row => row[11] === 'True'&& row[3] === 'SDSL' && row[4] === 'Pro' ).length;
       console.log('kpi3_11_pro_VDSL :', kpi3_11_pro_VDSL);
   
       const kpi_Pro_VDSL = data.slice(1).filter(row => row[11] === 'True' && row[5] === 'Clôturé' && row[7] === 'T' && row[3] === 'SDSL' && row[4] === 'Pro');
       const KPI2_6_Pro_VDSL= kpi_Pro_SDSL.length;
       console.log('kpi2.6 pro :', KPI2_6_Pro_SDSL);
   
       const { i_SDSL, j_SDSL, p_SDSL, m_SDSL } = kpi_Pro_SDSL.reduce((acc, row) => {
         const duration = row[2] - row[1];
   
         if (duration < 1) acc.i_SDSL++;
         else if (duration >= 1 && duration < 2) acc.j_SDSL++;
         else if (duration >= 2 && duration < 3) acc.p_SDSL++;
         else acc.m_SDSL++;
         return acc;
       }, { i_SDSL: 0, j_SDSL: 0, p_SDSL: 0, m_SDSL: 0 });
   
       const KPI2_2_Pro_SDSL = i_VDSL /KPI2_6_Pro_SDSL;
       console.log('kpi2.2 pro :', KPI2_2_Pro_SDSL);

       const KPI2_3_Pro_SDSL = j_SDSL /KPI2_6_Pro_SDSL ;
       console.log('kpi2.3 pro :', KPI2_3_Pro_SDSL);

       const KPI2_4_Pro_SDSL = p_SDSL /KPI2_6_Pro_SDSL ;
       console.log('kpi2.4 pro :', KPI2_4_Pro_SDSL);

       const KPI2_5_Pro_SDSL = m_SDSL  /KPI2_6_Pro_SDSL;
       console.log('kpi2.5 pro :', KPI2_5_Pro_SDSL);

   
       const kpi3_11_Res_SDSL = data.slice(1).filter(row => row[11] === 'True'&& row[3] === 'SDSL' && row[4] === 'Résidentiel' ).length;
       console.log('kpi3.11 Résidentiel :', kpi3_11_Res_SDSL);
       
        const kpi_Res_SDSL = data.slice(1).filter(row => row[11] === 'True' && row[5] === 'Clôturé' && row[7] === 'T' && row[3] === 'SDSL' && row[4] === 'Résidentiel');
         const KPI2_6_Res_VDSL = kpi_Res_SDSL.length;
           console.log('kpi2.6 Résidentiel :', KPI2_6_Res_SDSL);
   
       const { ir_SDSL, jr_SDSL, pr_SDSL, mr_SDSL } = kpi_Res_SDSL.reduce((acc, row) => {
         const duration = row[2] - row[1];
   
         if (duration < 1) acc.ir_SDSL++;
         else if (duration >= 1 && duration < 2) acc.jr_SDSL++;
         else if (duration >= 2 && duration < 3) acc.pr_SDSL++;
         else acc.mr_SDSL++;
         return acc;
       }, { ir_SDSL: 0, jr_SDSL: 0, pr_SDSL: 0, mr_SDSL: 0 });
   
       const KPI2_2_Res_SDSL = ir_SDSL / KPI2_6_Res_SDSL;
       console.log('kpi2.2 Résidentiel :', KPI2_2_Res_SDSL);

       const KPI2_3_Res_SDSL = jr_SDSL / KPI2_6_Res_SDSL;
       console.log('kpi2.3 Résidentiel :', KPI2_3_Res_SDSL);

       const KPI2_4_Res_SDSL = pr_SDSL / KPI2_6_Res_SDSL;
       console.log('kpi2.4 Résidentiel :', KPI2_4_Res_VDSL);

       const KPI2_5_Res_SDSL = mr_SDSL / KPI2_6_Res_SDSL;
       console.log('kpi2.5 Résidentiel :', KPI2_5_Res_SDSL);


      const insertQuery_VDSL = `INSERT INTO crm (
          idfsi, Mois, Technologie, KPI3_11_Pro, KPI3_11_Res, KPI2_1_Pro, 
          KPI2_1_Res, KPI2_2_Pro, KPI2_2_Res, KPI2_3_Pro, KPI2_3_Res, KPI2_4_Pro, 
          KPI2_4_Res, KPI2_5_Pro, KPI2_5_Res, KPI2_6_Pro, KPI2_6_Res
      ) VALUES (?, ?, ?, ?, ?, ?, ?, ?, ?, ?, ?, ?, ?, ?, ?, ?, ?)`;

      
      const values_VDSL = [username, mois, 'SDSL', kpi3_11_pro_SDSL, kpi3_11_Res_SDSL, 0, 0, KPI2_2_Pro_SDSL, KPI2_2_Res_SDSL, KPI2_3_Pro_SDSL, KPI2_3_Res_SDSL, KPI2_4_Pro_SDSL, KPI2_4_Res_SDSL, KPI2_5_Pro_SDSL, KPI2_5_Res_SDSL, KPI2_6_Pro_SDSL, KPI2_6_Res_SDSL];
      await new Promise((resolve, reject) => {
          connection.query(insertQuery_VDSL, values_VDSL, (err, result) => {
              if (err) {
                  console.error('Erreur lors de l\'insertion de l\'utilisateur :', err);
                  return reject(err);
              }
              resolve(result);
          });
      });

      res.status(200).json({ message: 'Données insérées avec succès' });

  } catch (error) {
      console.error('Erreur serveur :', error);
      res.status(500).json({ message: 'Erreur serveur', error: error.message });
  }
});
///////////////////fin sans function








/////// correcte avec function
const processData = (data, technology, type) => {
    const filteredData = data.slice(1).filter(row => row[11] === 'True' && row[3] === technology && row[4] === type);
    const kpi3_11 = filteredData.length;
    console.log('kpi3_11 :', kpi3_11);
  
    const kpi = filteredData.filter(row => row[5] === 'Clôturé' && row[7] === 'T');
    const KPI2_6 = kpi.length;
    console.log('kpi2_6 :', KPI2_6);
  
  
    const { ik, jk, pk, mk } = kpi.reduce((acc, row) => {
      const duration = row[2] - row[1];
      if (duration < 1) acc.ik++;
      else if (duration >= 1 && duration < 2) acc.jk++;
      else if (duration >= 2 && duration < 3) acc.pk++;
      else acc.mk++;
      return acc; }, { ik: 0, jk: 0, pk: 0, mk: 0 });
  
    const KPI2_2 = ik / KPI2_6;
    console.log('kpi2_2:', KPI2_2);
    console.log('kpi2_2:', ik);
  
    const KPI2_3 = jk / KPI2_6;
    console.log('kpi2_3 :', KPI2_3);
    console.log('kpi2_3:', jk);
  
    const KPI2_4 = pk / KPI2_6;
    console.log('kpi2_4 :', KPI2_4);
    console.log('kpi2_4:', pk);
  
  
    const KPI2_5 = mk / KPI2_6;
    console.log('kpi2_5 :', KPI2_2);
    console.log('kpi2_4:', mk);
  
  
    return { kpi3_11, KPI2_6, KPI2_2, KPI2_3, KPI2_4, KPI2_5 };
  };
  const insertData = (connection, username, mois, technology, proData, resData) => {
    const insertQuery = `INSERT INTO crm (
      idfsi, Mois, Technologie, KPI3_11_Pro, KPI3_11_Res, KPI2_1_Pro, 
      KPI2_1_Res, KPI2_2_Pro, KPI2_2_Res, KPI2_3_Pro, KPI2_3_Res, KPI2_4_Pro, 
      KPI2_4_Res, KPI2_5_Pro, KPI2_5_Res, KPI2_6_Pro, KPI2_6_Res
    ) VALUES (?, ?, ?, ?, ?, ?, ?, ?, ?, ?, ?, ?, ?, ?, ?, ?, ?)`;
  
    const values = [
      username, mois, technology, proData.kpi3_11, resData.kpi3_11, 0, 0, 
      proData.KPI2_2, resData.KPI2_2, proData.KPI2_3, resData.KPI2_3, 
      proData.KPI2_4, resData.KPI2_4, proData.KPI2_5, resData.KPI2_5, 
      proData.KPI2_6, resData.KPI2_6 ];
  
    connection.query(insertQuery, values, (err, result) => {
      if (err) {
        console.error('Erreur lors de l\'insertion de l\'utilisateur :', err);
        return res.status(500).json({ message: 'Erreur serveur' });
      }
      res.status(200).json({ message: 'Données insérées avec succès' });
    });
  };
  
  const storage2 = multer.diskStorage({
    destination: function (req, file, cb) { cb(null, path.join(__dirname, 'uploads2')); },
    filename: function (req, file, cb) { cb(null, file.originalname);  }
  });
  
  
  
  const upload2 = multer({ storage: storage2 });



app.post('/upload2', upload2.single('excelFile'), async (req, res) => {
  try {
      // Lecture du fichier Excel
      const workbook = xlsx.readFile(req.file.path);
      const sheetName = workbook.SheetNames[0];
      const sheet = workbook.Sheets[sheetName];
      const data = xlsx.utils.sheet_to_json(sheet, { header: 1 });
      const username = req.body.username;

      // Requête SQL pour insérer les données dans votre_table2
      const sql = `INSERT INTO votre_table2 (
          id_contrat, date_ouverture, date_cloture, type_offre, type_client, etat_ticket, 
          sujet, type_incident, indisponibilite, derangement, facturation, 
          internet_fixe, id_client, MOIS, Annee, userId	
      ) VALUES ?`;

      const values = data.slice(1).map(row => [...row.slice(0, 15), username]);

      await new Promise((resolve, reject) => {
          connection.query(sql, [values], (err, result) => {
              if (err) {
                  console.error('Erreur MySQL :', err.message);
                  return reject(err);
              }
              console.log('Données insérées avec succès :', result);
              resolve(result);
          });
      });

      // Calcul des KPI
      const mois = data[1][13];
      console.log('mois :', mois);
      const technologies = ['SDSL', 'ADSL', 'VDSL', 'FTTH'];

      technologies.forEach(technology => {
        console.log( '********',technology );

        console.log('Pro');

        const proData = processData(data, technology, 'Pro');
        console.log('Résidentiel');

        const resData = processData(data, technology, 'Résidentiel');
        insertData(connection, username, mois, technology, proData, resData);
      });

      res.status(200).json({ message: 'Données insérées avec succès' });

  } catch (error) {
      console.error('Erreur serveur :', error);
      res.status(500).json({ message: 'Erreur serveur', error: error.message });
  }
});
////////////////////////fin avec function





































///////////////////////
const xlsx = require('xlsx');
const mysql = require('mysql');
const multer = require('multer');
const path = require('path');

// Configuration de Multer
const storage = multer.diskStorage({
    destination: './uploads/',
    filename: (req, file, cb) => {
        cb(null, file.fieldname + '-' + Date.now() + path.extname(file.originalname));
    }
});
const upload2 = multer({ storage });

// Connexion MySQL
const connection = mysql.createConnection({
    host: 'localhost',
    user: 'root',
    password: 'password',
    database: 'votre_db'
});

// Route pour l'upload de fichier
app.post('/upload2', upload2.single('excelFile'), (req, res) => {
    if (!req.file) {
        return res.status(400).send('Aucun fichier envoyé.');
    }

    try {
        // Lecture du fichier Excel
        const workbook = xlsx.readFile(req.file.path);
        const sheetName = workbook.SheetNames[0];
        const sheet = workbook.Sheets[sheetName];
        const data = xlsx.utils.sheet_to_json(sheet, { header: 1 });

        if (data.length < 2) {
            return res.status(400).send('Le fichier Excel est vide ou mal formaté.');
        }

        const username = req.body.username;

        // Conversion des dates Excel en format lisible
        function excelDateToJSDate(excelSerial) {
            const date = new Date((excelSerial - 25569) * 86400000);
            return date.toISOString().split('T')[0]; // Format YYYY-MM-DD
        }

        // Requête SQL
        const sql = `INSERT INTO votre_table2 (
            id_contrat, date_ouverture, date_cloture, type_offre, type_client, etat_ticket, 
            sujet, type_incident, indisponibilite, derangement, facturation, 
            internet_fixe, id_client, MOIS, Annee, userId	
        ) VALUES ?`;

        const values = data.slice(1).map(row => [
            row[0],  // id_contrat
            excelDateToJSDate(row[1]),  // date_ouverture
            excelDateToJSDate(row[2]),  // date_cloture
            row[3],  // type_offre
            row[4],  // type_client
            row[5],  // etat_ticket
            row[6],  // sujet
            row[7],  // type_incident
            row[8],  // indisponibilite
            row[9],  // derangement
            row[10], // facturation
            row[11], // internet_fixe
            row[12], // id_client
            row[13], // MOIS
            row[14], // Annee
            username  // userId
        ]);

        connection.query(sql, [values], (err, result) => {
            if (err) {
                console.error('Erreur MySQL :', err.message);
                return res.status(500).send('Erreur lors de l\'insertion : ' + err.message);
            }
            console.log('Données insérées avec succès :', result);
            res.status(200).send('Fichier traité et données insérées.');
        });

    } catch (error) {
        console.error('Erreur lors du traitement du fichier :', error.message);
        res.status(500).send('Erreur de traitement : ' + error.message);
    }
});



















